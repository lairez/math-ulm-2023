\section{Espace des translatées d'une fonction}
\label{sec:espace-des-transl}


Soit~$g \in \mathcal{C}(\mathbb{R})$ une fonction intégrable.
Pour~$A\subseteq \mathbb{Z}$, on note $\mathcal{S}_A$ le sous-espace vectoriel de~$\mathcal{C}(\mathbb{R})$ engendré par les fonctions $x\mapsto g(x-a)$, avec~$a\in A$.
On suppose que pour toute $f \in \mathcal{C}(\mathbb{R})$ intégrable et tout $\epsilon > 0$, il existe $h \in \mathcal{S}_\mathbb{Z}$ telle que
\[ \int_\mathbb{R} |f(x)-h(x)| \mathrm{d}x<\epsilon. \]

Montrer que pour toute $f\in \mathcal{C}(\mathbb{R})$ intégrable et tout $\epsilon >0$, il existe $L>0$ tel que
pour tout $y \in \mathbb{R}$, il existe~$A \subset \mathbb{Z}$ et~$h\in \mathcal{S}_A$ tels que
\[ \# A \leq L \text{ et } \int_\mathbb{R} |f(x-y)-h(x)|\mathrm{d} x< \epsilon. \]

\begin{em}
  L'énoncé est correct, mais l'hypothèse est vide, nous avons abusivement simplifié un vrai énoncé en replaçant une famille de fonctions par une seule. Il y a donc deux méthodes assez différentes pour résoudre l'exercice: montrer l'implication, ou bien montrer que l'hypothèse est vide.
\end{em}

%%% Local Variables:
%%% mode: latex
%%% TeX-master: "main"
%%% End:
