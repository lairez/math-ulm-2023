\section{Théorème de Peano}

Soit~$I \subseteq \mathbb{R}$ un intervalle ouvert non vide.
Soit~$\mathcal{C}^r(I)$ le~$\mathbb{R}$-espace vectoriel des fonctions sur~$I$
à valeurs réelles continument dérivables~$r$ fois. Pour toutes fonctions
$f_1,\dotsc,f_r \in \mathcal{C}^r(I)$
on définit une fonction~$I\to \mathbb{R}$
\[ \mathcal{W}[f_1,\dotsc,f_r](x) = \left|
    \begin{matrix}
      f_1 & \dotsb & f_r \\
      f_1' & \dotsb & f_r' \\
      \vdots & & \vdots \\
      f_1^{(r-1)} & \dotsb & f_r^{(r-1)}
    \end{matrix}
  \right|. \]

Soient~$f_1,\dotsc,f_r \in \mathcal{C}^r(I)$.
On note $W = \mathcal{W}[f_1,\dotsc,f_n]$
et, pour~$1\leq i\leq r$,
\[ V_i = (-1)^i \mathcal{W}[f_1,\dotsc,f_{i-1},f_{i+1},\dotsc,f_r]. \]
Montrer que si~$V_r$ ne s'annule pas sur~$I$ et si~$W \equiv 0$, alors les $f_1,\dotsc,f_r$ forment une famille liée.

On essaie ensuite de donner une démonstration directe, sans utiliser le théorème de Cauchy.
On pourra montrer que si~$W \equiv 0$, alors pour tout~$0\leq k< r$,
\[ \sum_{i=1}^r V_i(x) f_i^{(k)}(x) = 0 \quad\text{et}\quad  \sum_{i=1}^r V_i'(x) f_i^{(k)}(x) = 0. \]


\begin{em}
  C'est un théorème dû à \textcite{Peano_1889} mais, pour la dernière partie de l'énoncé, nous suivons la démonstration de \textcite{Bocher_1900}.
\end{em}


%%% Local Variables:
%%% mode: latex
%%% TeX-master: "main"
%%% End:
