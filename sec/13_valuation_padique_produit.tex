\section{Valuation $p$-adique d'un produit}

Soit $p$ et $q$ deux entier premiers distincts. Montrer qu'il existe une constante $c > 0$ (que l'on estimera) tel que pour tout entier~$m>0$, la valuation $p$-adique du produit
\[ N(m)=(q-1)(q^2-1)\ldots(q^m-1), \]
est majorée par $cm\log m$.

\begin{em}
  On peut seulement supposer que~$q$ est premier avec~$p$. Il est aussi possible d'obtenir une borne linéaire en~$m$, avec une analyse un peu plus fine.
\end{em}

%%% Local Variables:
%%% mode: latex
%%% TeX-master: "main"
%%% End:
