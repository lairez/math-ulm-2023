\section{Étude des sous-groupes des isométries affines}

Soit~$G$ un sous-groupe du groupe des isométries du plan affine~$\mathbb{R}^2$.
On suppose que pour tout~$x\in \mathbb{R}^2$, il existe~$g\in G$ tel que~$g(x)\neq x$.
Montrer que~$G$ contient une translation non triviale.

Si de plus~$G$ ne stabilise aucune droite, montrer que~$G$ contient une deuxième translation non parallèle à la première.

\begin{em}
  La classification des isométries affines est hors programme (et de fait peu de
  candidats la connaissait), mais ce n'est pas un prérequis pour l'exercice,
  surtout si on pense au formalisme complexe pour les similitudes (qui lui est
  au programme).
  On s'en sort aussi très bien en s'appuyant sur la classification des isométries vectorielles du plan (au programme).
  Un raisonnement purement géométrique est aussi possible.
  On peut aussi supposer que~$G$ ne contient que des isométries directes pour simplifier.
\end{em}

%%% Local Variables:
%%% mode: latex
%%% TeX-master: "main"
%%% End:
