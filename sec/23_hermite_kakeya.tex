
\section{Théorème d'Hermite--Kakeya}

Soient~$P$ et $Q \in \mathbb{R}[X]$ des polynômes non constants. On dit que~$P$
et~$Q$ \emph{s'entrelacent} si: (1)~leurs racines sont réelles et simples,
(2)~ils n'ont pas de racines réelles communes et (3)~entre deux racines
consécutives de~$Q$ (resp.~$P$), il y a une et une seule racine de~$P$
(resp.~$Q$).

Montre que si
pour tout~$(\lambda, \mu) \in \mathbb{R}^2 \setminus \left\{ (0,0) \right\}$,
les racines de $\lambda P + \mu Q$ sont toutes réelles et simples,
alors~$P$ et~$Q$ s'entrelacent.

Montrer la réciproque.

\begin{em}
  \textcite{RahmanSchmeisser_2002} donnent une démonstration, mais on peut faire plus élémentaire (surtout pour la réciproque).
\end{em}

%%% Local Variables:
%%% mode: latex
%%% TeX-master: "main"
%%% End:
