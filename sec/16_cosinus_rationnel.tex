\section{Valeurs rationnelles du cosinus}

Décrire l'ensemble des nombres rationels~$r$ tels que~$\cos(r \pi)$ est rationel.

\begin{em}
  C'est plus classique que ce qui nous aurions aimé.
  La plupart des candidats ont pensé à l'approche utilisant la réduction~$\cos(2x) = 2\cos(x)^2 - 1$.
  Nous avions en tête une autre approche utilisant le lemme suivant, que l'on peut montrer sans la machinerie des entiers algébriques.
  Soit~$\mu \in \mathbb{Z}[X]$ un polynôme unitaire, et soit~$\alpha\in \mathbb{C}$ une racine de~$\mu$.
  Pour tout~$g\in \mathbb{Z}[X]$, $g(\alpha)$ est racine d'un polynôme unitaire à coefficients entiers.
\end{em}

%%% Local Variables:
%%% mode: latex
%%% TeX-master: "main"
%%% End:
