\section{Minimisation locale sur un graphe}

Soit~$E$ un ensemble fini et~$V : E \to \mathcal{P}(E)$ une fonction de~$E$ vers les parties de~$E$.
Soit~$f : E \to \mathbb{R}$ une fonction.
Un point~$a \in E$ est un \emph{minimum local} si
$f(a) \leq f(b)$ pour tout~$b \in V(a)$.

Soit~$M$ un entier tel que~$M \geq \sqrt{\# E}$. Soient~$b_1,\dotsc,b_M$ des variables aléatoires indépendantes et uniformément distribuées dans~$E$.
Soit~$k$ tel que $f(b_k) = \min_{1\leq i \leq M} f(b_i)$.

Soit~$(u_n)_{n\geq 0}$ une suite de~$E$ telle que~$u_0 = b_k$
et pour tout~$n \geq 0$:
\begin{itemize}
  \item si $u_n$ est un minimum local, alors~$u_{n+1} = u_n$;
  \item sinon, $u_{n+1} \in V(u_n)$ et~$f(u_{n+1}) < f(u_n)$.
\end{itemize}

Montrer que $u_{M}$ est un minimum local avec probabilité au moins~$1/2$.

\begin{em}
  Inspiré par \textcite{Aldous_1983}, même s'il considère cet énoncé précis comme trivial.
\end{em}

%%% Local Variables:
%%% mode: latex
%%% TeX-master: "main"
%%% End:
