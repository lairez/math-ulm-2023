\documentclass[a4paper,parskip=half-,headings=small]{scrartcl}

\usepackage{libertine}
\usepackage[T1]{fontenc}
\usepackage[libertine,vvarbb]{newtxmath}
\usepackage{microtype}
\usepackage[hypertexnames=false]{hyperref}
\usepackage[french]{babel}
\usepackage[scale=.85]{plex-sans}

\usepackage{mathtools}
\usepackage[shortlabels]{enumitem}

\usepackage[style=apa, backend=biber]{biblatex}
\addbibresource[glob]{*.bib}

\addtokomafont{author}{\sffamily}
\addtokomafont{date}{\sffamily}

\renewcommand{\leq}{\leqslant}
\renewcommand{\geq}{\geqslant}
\renewcommand{\epsilon}{\varepsilon}
\renewcommand{\phi}{\varphi}

\newcommand{\st}{\ \middle|\ }

\usepackage{fontawesome}
\usepackage{graphicx}

\title{Épreuve orale “Mathématiques Ulm”}
\subtitle{Concours MP --- Session 2023}
\author{Pierre Lairez \and Romain Tessera}
\date{}

\begin{document}

\maketitle
\begin{center}
  \begin{minipage}{.7\linewidth}
    \footnotesize\sffamily
    Ce document rassemble les exercices posés aux candidats au concours d'entrée de l'ENS Ulm, en filière MP, lors de l'épreuve orale “Mathématiques Ulm”.
    \bigskip

    \faGithub{} \url{https://github.com/lairez/math-ulm-2023}

    \bigskip
    \raisebox{-.4ex}{\includegraphics[height=2ex]{cc-zero.pdf}}
    Document publié sous licence \href{https://creativecommons.org/publicdomain/zero/1.0/deed.fr}{``CC0 1.0 Universel''}.
  \end{minipage}
\end{center}

\section{Wronskiens et systèmes de Tchebychev}

Soit~$I \subseteq \mathbb{R}$ un intervalle ouvert non vide.
Soit~$\mathcal{C}^r(I)$ le~$\mathbb{R}$-espace vectoriel des fonctions sur~$I$
à valeurs réelles continument dérivables $r$ fois. Pour toutes fonctions
$f_1,\dotsc,f_r \in \mathcal{C}^{r-1}(I)$
on définit une fonction~$I\to \mathbb{R}$
\[ \mathcal{W}[f_1,\dotsc,f_r](x) = \left|
    \begin{matrix}
      f_1 & \dotsb & f_r \\
      f_1' & \dotsb & f_r' \\
      \vdots & & \vdots \\
      f_1^{(r-1)} & \dotsb & f_r^{(r-1)}
    \end{matrix}
  \right|. \]

\begin{enumerate}
  \item Montrer que pour toute fonctions~$g,f_1,\dotsc,f_r \in C^{r-1}(I)$,
        \[ \mathcal{W}[gf_1,\dotsc,g f_r](x) = g(x)^r \mathcal{W}[f_1,\dotsc,f_r](x). \]


  \item Soit~$f_1,\dotsc,f_r \in C^{r-1}(I)$ telles que~$\mathcal{W}[f_1,\dotsc,f_k]$ est strictement positif sur~$I$, pour tout~$1\leq k\leq r$.
        Montrer que pour tout~$a_1,\dotsc,a_r \in \mathbb{R}$ non tous nuls, la fonction~$a_1 f_1 + \dotsb + a_r f_r$ admet au plus~$r-1$ zéros sur~$I$.
\end{enumerate}

\begin{em}
  Pour la seconde question, le cas $r=2$ est instructif.
  C'est un théorème classique qui admet aussi une réciproque \parencite[Chap.~XI, Théorème~1.2]{KarlinStudden_1966}
\end{em}

%%% Local Variables:
%%% mode: latex
%%% TeX-master: "main"
%%% End:

\section{Une propriété de divisibilité du cardinal des matrices inversibles modulo~$p$}

Soit $p$ un entier premier impair, et $n\geq 3$ un entier. Montrer que $n$
divise le cardinal du groupe $\operatorname{GL}_{n-1}(\mathbb{Z}/p\mathbb{Z})$ des
matrices inversibles de taille~$n-1$ à coefficients
dans~$\mathbb{Z}/p\mathbb{Z}$.

\begin{em}
  La restriction à~$p$ impair n'est pas nécessaire, mais elle peut simplifier
  certains calculs.
  La formule du cardinal n'est pas au programme mais c'est un classique.
  Pour ceux qui ne la connaissait pas, l'exercice est intéressant.
  On pouvait aussi s'en passer (voir la solution algébrique).
\end{em}

%%% Local Variables:
%%% mode: latex
%%% TeX-master: "main"
%%% End:

\section{Minimisation locale sur un graphe}

Soit~$E$ un ensemble fini et~$V : E \to \mathcal{P}(E)$ une fonction de~$E$ vers les parties de~$E$.
Soit~$f : E \to \mathbb{R}$ une fonction.
Un point~$a \in E$ est un \emph{minimum local} si
$f(a) \leq f(b)$ pour tout~$b \in V(a)$.

Soit~$M$ un entier tel que~$M \geq \sqrt{\# E}$. Soient~$b_1,\dotsc,b_M$ des variables aléatoires indépendantes et uniformément distribuées dans~$E$.
Soit~$k$ tel que $f(b_k) = \min_{1\leq i \leq M} f(b_i)$.

Soit~$(u_n)_{n\geq 0}$ une suite de~$E$ telle que~$u_0 = b_k$
et pour tout~$n \geq 0$:
\begin{itemize}
  \item si $u_n$ est un minimum local, alors~$u_{n+1} = u_n$;
  \item sinon, $u_{n+1} \in V(u_n)$ et~$f(u_{n+1}) < f(u_n)$.
\end{itemize}

Montrer que $u_{M}$ est un minimum local avec probabilité au moins~$1/2$.

\begin{em}
  Inspiré par \textcite{Aldous_1983}, même s'il considère cet énoncé précis comme trivial.
\end{em}

%%% Local Variables:
%%% mode: latex
%%% TeX-master: "main"
%%% End:

\section{Espace des translatées d'une fonction}
\label{sec:espace-des-transl}


Soit~$g \in \mathcal{C}(\mathbb{R})$ une fonction intégrable.
Pour~$A\subseteq \mathbb{Z}$, on note $\mathcal{S}_A$ le sous-espace vectoriel de~$\mathcal{C}(\mathbb{R})$ engendré par les fonctions $x\mapsto g(x-a)$, avec~$a\in A$.
On suppose que pour toute $f \in \mathcal{C}(\mathbb{R})$ intégrable et tout $\epsilon > 0$, il existe $h \in \mathcal{S}_\mathbb{Z}$ telle que
\[ \int_\mathbb{R} |f(x)-h(x)| \mathrm{d}x<\epsilon. \]

Montrer que pour toute $f\in \mathcal{C}(\mathbb{R})$ intégrable et tout $\epsilon >0$, il existe $L>0$ tel que
pour tout $y \in \mathbb{R}$, il existe~$A \subset \mathbb{Z}$ et~$h\in \mathcal{S}_A$ tels que
\[ \# A \leq L \text{ et } \int_\mathbb{R} |f(x-y)-h(x)|\mathrm{d} x< \epsilon. \]

\begin{em}
  L'énoncé est correct, mais l'hypothèse est vide, nous avons abusivement simplifié un énoncé véritablement intéressant en replaçant une famille de fonctions par une seule. Il y a donc deux méthodes assez différentes pour résoudre l'exercice: montrer l'implication, ou bien montrer que l'hypothèse est vide.
\end{em}

%%% Local Variables:
%%% mode: latex
%%% TeX-master: "main"
%%% End:

\section{Limite d'une série alternée}

Soit~$f \in \mathcal{C}^1(\mathbb{R})$ décroissante et tendant vers~0 en~$+\infty$.
Montrer que la fonction
\[ g(x) = \sum_{n = 0}^\infty (-1)^n f(nx) \]
est bien définie pour~$x > 0$.
Donner sa limite en~0.

%%% Local Variables:
%%% mode: latex
%%% TeX-master: "main"
%%% End:

\section{Unions de fermés}

\def\zuo{\rbrack 0,1\lbrack}
Montrer que $\zuo$ n'est pas l'union d'un nombre dénom\-brable d'inter\-valles fermés disjoints d'intérieur non vide.

Montrer que le carré ouvert~$\zuo^2$ n'est pas l'union d'un numbre dénombrable de disques fermés.

\begin{em}
  Le premier énoncé est un cas particulier d'un théorème plus général de \textcite{Sierpinski_1918}. Le second découle du premier assez directement, mais on note aussi la généralisation due à \textcite{Dijkstra_1984}.
\end{em}

%%% Local Variables:
%%% mode: latex
%%% TeX-master: "main"
%%% End:

\section{Une inégalité isopérimétrique discrète}

Soient~$n$ et~$d$ des entiers strictement positifs et~$G = \left( \mathbb{Z}/n \mathbb{Z} \right)^d$.
Soit~$S = \left\{ \pm e_1,\dotsc,\pm e_d \right\}$, où~$e_i = (0,\dotsc,0,1,0,\dotsc,0) \in G$, avec le~1 en~$i$\textsuperscript{e} position.
Soient~$X$ une variable aléatoire uniformément distribuée dans~$G$ et~$f : G\to \mathbb{R}$ une fonction.
Montrer que
\[ \mathbb{E} \left[ \left| f(X) - \mathbb{E}[f(X)] \right| \right] \leq \frac{dn}{2} \max_{s\in S} \mathbb{E} \left[ |f(X) - f(X + s)| \right]. \]

Soit~$X$ l'ensemble des sommets de l'hypercube~$[0,1]^d$.
Soit~$A \subset X$ un sous-ensemble strict non vide.
Soit~$n_A$ le nombre d'arrêtes de~$X$ ayant une et une seule extrémité dans~$A$.
Donner un majorant de
\[ \frac{\operatorname{card}(A) \operatorname{card}(X\setminus A)}{n_A}. \]


%%% Local Variables:
%%% mode: latex
%%% TeX-master: "main"
%%% End:

\section{Une caractérisation des matrices antisymétriques}

Soit~$n$ entier positif impair.
Soit~$A \in \mathcal{M}_{n}(\mathbb{R})$ telle pour toute matrice antisymétrique~$M\in \mathcal{M}_{n}(\mathbb{R})$, $\det(A+M) = 0$. Montrer que~$A$ est antisymétrique.


%%% Local Variables:
%%% mode: latex
%%% TeX-master: "main"
%%% End:

\section{Étude des sous-groupes des isométries affines}

Soit~$G$ un sous-groupe du groupe des isométries du plan affine~$\mathbb{R}^2$.
On suppose que pour tout~$x\in \mathbb{R}^2$, il existe~$g\in G$ tel que~$g(x)\neq x$.
Montrer que~$G$ contient une translation non triviale.

Si de plus~$G$ ne stabilise aucune droite, montrer que~$G$ contient une deuxième translation non parallèle à la première.

\begin{em}
  La classification des isométries affines est hors programme (et de fait peu de
  candidats la connaissait), mais ce n'est pas un prérequis pour l'exercice,
  surtout si on pense au formalisme complexe pour les similitudes (qui lui est
  au programme).
  On s'en sort aussi très bien en s'appuyant sur la classification des isométries vectorielles du plan (au programme).
  Un raisonnement purement géométrique est aussi possible.
  On peut aussi supposer que~$G$ ne contient que des isométries directes pour simplifier.
\end{em}

%%% Local Variables:
%%% mode: latex
%%% TeX-master: "main"
%%% End:

\section{Résultant}

Soient~$A, B \in \mathbb{C}[X]$ deux polynômes unitaires, de degrés respectifs~$a$ et~$b$.
Soit~$M_{A,B}$ l'endomorphisme de~$\mathbb{C}[X]/(A)$ défini par~$M_{A,B}([P]) = [BP]$.
Soit~$\mu_{A, B} = \det M_{A,B}$.
Montrer que~$\mu_{A, B} = (-1)^{a b} \mu_{B, A}$.

Soit~$F_{A, B}$ l'unique endomorphisme de~$\mathbb{C}[X]_{a+b-1}$
tel que pour tout~$U \in \mathbb{C}[X]_{a-1}$ et tout~$V\in \mathbb{C}[X]_{b-1}$,
$F_{A,B}(U + X^a V) = BU + AV$.
Montrer que $\det(F_{A, B}) = \mu_{A,B}$

\begin{em}
  Le nombre~$\mu_{A,B}$ est le \emph{résultant} de~$A$ et de~$B$.
\end{em}

%%% Local Variables:
%%% mode: latex
%%% TeX-master: "main"
%%% End:


\section{Composantes connexes d'ensembles de polynômes}

Soit~$d\geq 1$ un entier.
Soit~$P$ l'ensemble des polynômes unitaires de degré~$d$ à coefficients réels.

Décrire les composantes connexes par arcs de
\[ \left\{ (f, x) \in P \times \mathbb{R} \ \middle|\ f(x) = 0 \text{ et } f'(x) \neq 0 \right\}. \]

Décrire les composantes connexes par arcs de
\[ \left\{ f\in P \ \middle|\ \forall x\in \mathbb{R}, f(x) \neq 0 \text{ ou } f'(x) \neq 0 \right\}. \]

%%% Local Variables:
%%% mode: latex
%%% TeX-master: "main"
%%% End:

\section{Impossibilité de la densité d'un certain espace de translations}

Soit  $B(\mathbb{R})$  l'espace vectoriel des fonctions bornées sur $\mathbb{R}$ muni de la norme uniforme.
Soit $g : \mathbb{R}\to \mathbb{R}$ une fonction à support compact. On note $W(g)\subseteq B(\mathbb{R})$ l'espace engendré par les translatés $x\mapsto g(x -n)$ pour $n\in \mathbb{Z}$.
Etudier l'ensemble  $t\in \mathbb{R}$ tels que $\overline{W(g)}$ est invariant par translation par $t$.

\begin{em}
  Il s'agit de montrer que si~$g$ est non nulle, alors l'ensemble en question est un sous-groupe discret de~$\mathbb{R}$.
  L'énoncé est inspiré par l'hypothèse vide de l'exercice~\ref{sec:espace-des-transl}.
\end{em}

%%% Local Variables:
%%% mode: latex
%%% TeX-master: "main"
%%% End:

\section{Valuation $p$-adique d'un produit}

Soit $p$ et $q$ deux entier premiers distincts. Montrer qu'il existe une constante $c > 0$ (que l'on estimera) tel que pour tout entier~$m>0$, la valuation $p$-adique du produit
\[ N(m)=(q-1)(q^2-1)\ldots(q^m-1), \]
est majorée par $cm\log m$.

\begin{em}
  On peut seulement supposer que~$q$ est premier avec~$p$. Il est aussi possible d'obtenir une borne linéaire en~$m$, avec une analyse un peu plus fine.
\end{em}

%%% Local Variables:
%%% mode: latex
%%% TeX-master: "main"
%%% End:


\section{Générateurs d'un groupe de matrices}

Soit~$G$ le sous-ensemble de~$\operatorname{GL}_2(\mathbb{Z})$
des matrices à coefficients entiers~$ \begin{psmallmatrix} a&b\\c&d \end{psmallmatrix}$ telles que
$a\equiv d \equiv 1-c \equiv 1 \pmod{3}$ et~$ad-bc=1$.

Montrer que~$G$ est un sous-groupe engendré par
$A = \begin{psmallmatrix} 1&1\\0&1 \end{psmallmatrix}$
et~$B = \begin{psmallmatrix} 1&0\\3&1 \end{psmallmatrix}$.

\begin{em}
  C'est une variante, à peine plus subtile, d'un résultat analogue sur~$\operatorname{SL}_2(\mathbb{Z})$.
\end{em}

%%% Local Variables:
%%% mode: latex
%%% TeX-master: "main"
%%% End:

\section{Angles d'un pavage}

Soit $G = \left\{ z\to az+b \st a, b \in \mathbb{C}, |a| = 1 \right\}$. C'est un sous-groupe du groupe des bijections~$\mathbb{C}\to\mathbb{C}$ muni de la composition.
Soit~$H \subseteq G$ un sous-groupe contenant deux translations selon des vecteurs~$b_1,b_2 \in \mathbb{C}$ formant une famille libre sur~$\mathbb{R}$.
On suppose de plus que pour tout~$h \in H$, soit~$h(0) = 0$, soit~$|h(0)| \geq 1$.
Montrer que l'ensemble $\left\{ h'(0) \st h\in H \right\}$ est fini.

Montrer que le cardinal de cet ensemble divise~6.

\begin{em}
  C'est l'un des ingrédients de la classification des pavages du plan.
  L'existence de deux translations n'est pas nécessaire, une seule suffit, mais en donner deux pouvait rendre l'énoncé plus accessible.
\end{em}

%%% Local Variables:
%%% mode: latex
%%% TeX-master: "main"
%%% End:

\section{Valeurs rationnelles du cosinus}

Décrire l'ensemble des nombres rationels~$r$ tels que~$\cos(r \pi)$ est rationel.

\begin{em}
  C'est plus classique que ce qui nous aurions aimé.
  La plupart des candidats ont pensé à l'approche utilisant la réduction~$\cos(2x) = 2\cos(x)^2 - 1$.
  Nous avions en tête une autre approche utilisant le lemme suivant, que l'on peut montrer sans la machinerie des entiers algébriques.
  Soit~$\mu \in \mathbb{Z}[X]$ un polynôme unitaire, et soit~$\alpha\in \mathbb{C}$ une racine de~$\mu$.
  Pour tout~$g\in \mathbb{Z}[X]$, $g(\alpha)$ est racine d'un polynôme unitaire à coefficients entiers.
\end{em}

%%% Local Variables:
%%% mode: latex
%%% TeX-master: "main"
%%% End:

\section{Théorème de Peano}

Soit~$I \subseteq \mathbb{R}$ un intervalle ouvert non vide.
Soit~$\mathcal{C}^r(I)$ le~$\mathbb{R}$-espace vectoriel des fonctions sur~$I$
à valeurs réelles continument dérivables~$r$ fois. Pour toutes fonctions
$f_1,\dotsc,f_r \in \mathcal{C}^r(I)$
on définit une fonction~$I\to \mathbb{R}$
\[ \mathcal{W}[f_1,\dotsc,f_r](x) = \left|
    \begin{matrix}
      f_1 & \dotsb & f_r \\
      f_1' & \dotsb & f_r' \\
      \vdots & & \vdots \\
      f_1^{(r-1)} & \dotsb & f_r^{(r-1)}
    \end{matrix}
  \right|. \]

Soient~$f_1,\dotsc,f_r \in \mathcal{C}^r(I)$.
On note $W = \mathcal{W}[f_1,\dotsc,f_n]$
et, pour~$1\leq i\leq r$,
\[ V_i = (-1)^i \mathcal{W}[f_1,\dotsc,f_{i-1},f_{i+1},\dotsc,f_r]. \]
Montrer que si~$V_r$ ne s'annule pas sur~$I$ et si~$W \equiv 0$, alors les $f_1,\dotsc,f_r$ forment une famille liée.

On essaie ensuite de donner une démonstration directe, sans utiliser le théorème de Cauchy.
On pourra montrer que si~$W \equiv 0$, alors pour tout~$0\leq k< r$,
\[ \sum_{i=1}^r V_i(x) f_i^{(k)}(x) = 0 \quad\text{et}\quad  \sum_{i=1}^r V_i'(x) f_i^{(k)}(x) = 0. \]


\begin{em}
  C'est un théorème dû à \textcite{Peano_1889} mais, pour la dernière partie de l'énoncé, nous suivons la démonstration de \textcite{Bocher_1900}.
\end{em}


%%% Local Variables:
%%% mode: latex
%%% TeX-master: "main"
%%% End:

\section{Une distance sur les matrices symétriques}

On note $\mathcal{S}^{++}_n$ l'ensemble des matrices réelles symétriques de taille $n$ définies positives.
Montrer que pour toute paire $A,B\in \mathcal{S}^{++}_n$, il existe $G\in \operatorname{GL}_n(\mathbb{R})$ tel que $B=GAG^t$.

Pour toute fonction $f:\mathbb{R}_+^*\to \mathbb{R}$ et $A\in \mathcal{S}^{++}_n$, donner un sens à $f(A)$. À l'aide de cette définition, on pose
\[d(A,B)=\| \log (A^{-1/2}BA^{-1/2}) \|,\]
où~$\|-\|$ est la norme d'opérateur relative à la norme euclidienne sur~$\mathbb{R}^n$.
Montrer que
\[ d(GAG^t,GBG^t)=d(A,B) \]
pour tout $G\in \operatorname{GL}_n(\mathbb{R})$, puis que~$d$ définit une distance sur~$\mathcal{S}^{++}_n$.

%%% Local Variables:
%%% mode: latex
%%% TeX-master: "main"
%%% End:

\section{Norme de l'inverse d'une matrice à lignes unitaires}

Soit $A$ une matrice réelle carrée de taille $n \geq 1$
donc les lignes~$L_1,\dotsc,L_n$ sont des vecteurs unitaires.
Soit~$\epsilon > 0$ tel que, pour tout~$1\leq i\leq n$, la distance euclidienne
de $L_i$ au sous espace engendré par les $L_j$, avec $j\neq i$, est minorée par $\epsilon$.

Montrer que~$A$ est inversible et que $\|A^{-1}x\|_2\leq \epsilon^{-1} \|x\|_1$, pour tout~$x\in \mathbb{R}^n$,
où $\|x\|_1=\sum_i|x_i|$ et $\|x\|_2^2=\sum_i x_i^2$.

%%% Local Variables:
%%% mode: latex
%%% TeX-master: "main"
%%% End:

\section{Disques et carrés}

Soit~$D$ le disque fermé de centre 0 et rayon~1 dans~$\mathbb{R}^2$. Montrer qu'il existe une suite~$C_0, C_1, \dotsc$ de carrés de $\mathbb{R}^2$
tels que:
\begin{enumerate}[(i)]
  \item $\forall i \geq0, C_i \subseteq D$;
  \item $\forall i, j \geq0: i\neq j \Rightarrow \mathring C_i \cap \mathring C_j = \varnothing$;
  \item $\sum_{i \geq 0} \operatorname{Aire}(C_i) = \pi$.
\end{enumerate}

Soit~$C = [-1,1]^2$.
Montrer qu'il existe une suite~$D_0, D_1, \dotsc$ de disques de $\mathbb{R}^2$
tels que:
\begin{enumerate}[(i)]
  \item $\forall i \geq0, D_i \subseteq C$;
  \item $\forall i, j \geq0: i\neq j \Rightarrow \mathring D_i \cap \mathring D_j = \varnothing$;
  \item $\sum_{i \geq 0} \operatorname{Aire}(D_i) = 4$.
\end{enumerate}


%%% Local Variables:
%%% mode: latex
%%% TeX-master: "main"
%%% End:

\section{Certification de racines}

Soit~$f : \mathbb{R}^n\to \mathbb{R}^n$ une application $\mathcal{C}^1$.
Soit~$x \in \mathbb{R}^n$, soit~$B$ la boule unité fermée.
On suppose que pour tout~$u, v \in B$,
\[ - f(x) + v - \mathrm{d}f(x+u) \cdot v \in \tfrac 12 B. \]
Montrer que~$f$ admet un unique zéro dans la boule~$x + B$.

\begin{em}
  C'est un critère dû à \textcite{Krawczyk_1969}, voir aussi \textcite{Rump_1983}.
  L'approche usuelle pour la démonstration utilise un argument de point fixe, mais la constante~$\frac12$ permet d'autres approches.
\end{em}

%%% Local Variables:
%%% mode: latex
%%% TeX-master: "main"
%%% End:

\section{Médiane de moyennes}

Soient~$n, m \geq 1$ des entiers
et soient~$X_{i,j}$, pour~$1\leq i \leq n$ et~$1\leq j \leq m$, des variables aléatoires discrètes i.i.d. de variance~$\sigma^2$ et de moyenne~$\mu$.
Pour~$1\leq i \leq n$, soit~$Y_i = \frac{1}{m}\sum_{j=1}^m X_{i,j}$.
Soit~$Z$ une médiane  de l'ensemble $\left\{ Y_1,\dotsc,Y_n \right\}$.

Montrer que
\[ \mathbb{P} \left[ |Z-\mu| \leq \frac{2\sigma}{\sqrt{m}} \right] \geq 1 - \left( \frac 34 \right)^{\frac n2}. \]

%%% Local Variables:
%%% mode: latex
%%% TeX-master: "main"
%%% End:

\section{Sous-espace stable}

Soit~$V$ un espace vectoriel normé de dimension finie.
On considère deux endomorphismes de~$V$, notés $h_1$ et~$h_2$, préservant la norme, et tels que $h_1$ et $h_2$ commutent avec leur commutateur~$h_1h_2h_1^{-1}h_2^{-1}$.
Montrer que le sous-espace des vecteurs invariants par~$h_1$ et $h_2$ admet un supplémentaire également stable par $h_1$ et $h_2$.

%%% Local Variables:
%%% mode: latex
%%% TeX-master: "main"
%%% End:


\section{Théorème d'Hermite--Kakeya}

Soient~$P$ et $Q \in \mathbb{R}[X]$ des polynômes non constants. On dit que~$P$
et~$Q$ \emph{s'entrelacent} si: (1)~leurs racines sont réelles et simples,
(2)~ils n'ont pas de racines réelles communes et (3)~entre deux racines
consécutives de~$Q$ (resp.~$P$), il y a une et une seule racine de~$P$
(resp.~$Q$).

Montre que si
pour tout~$(\lambda, \mu) \in \mathbb{R}^2 \setminus \left\{ (0,0) \right\}$,
les racines de $\lambda P + \mu Q$ sont toutes réelles et simples,
alors~$P$ et~$Q$ s'entrelacent.

Montrer la réciproque.

\begin{em}
  \textcite{RahmanSchmeisser_2002} donnent une démonstration, mais on peut faire plus élémentaire (surtout pour la réciproque).
\end{em}

%%% Local Variables:
%%% mode: latex
%%% TeX-master: "main"
%%% End:


\section{Un groupe de polynômes}

Soit~$p$ un nombre premier.
On considère l'anneau  $A$ des fractions rationnelles en $X$ à coefficients dans $\mathbb{Z}/p^2 \mathbb{Z}$ de la forme~$X^{-k} P(X)$, avec~$P$ un polynôme.
Décrire le groupe $A^\times$ des éléments inversibles (pour la multiplication) et montrer qu'il n'est pas engendré par un nombre fini d'éléments.

%%% Local Variables:
%%% mode: latex
%%% TeX-master: "main"
%%% End:


\section{Matrices de traces nulles et sommes de deux carrés}

Soient~$A, B \in \operatorname{SL}_2(\mathbb{Z})$ (le groupes des matrices $2\times 2$ à coefficients entiers et déterminant~1).
Montrer que si~$\operatorname{Tr}(A) = \operatorname{Tr}(B) = 0$, alors~$A$ est conjuguée à~$B$ ou~$-B$.

Montrer que si~$n > 1$ et~$x$ sont des entiers tels que~$x^2 \equiv -1 \pmod{n}$, alors il existe des entiers~$a$ et~$b$ tels que~$n = a^2 + b^2$.

%%% Local Variables:
%%% mode: latex
%%% TeX-master: "main"
%%% End:


\section{Théorème d'Hermite--Sylvester}


Soit~$P \in \mathbb{R}[X]$ un polynôme de degré~$n \geq 1$. Soit~$\lambda_1,\dotsc,\lambda_n \in \mathbb{C}$ ses racines, avec multiplicité.
Soit~$H \in \mathbb{C}^{n\times n}$ la matrice définie par
\[ H_{i,j} = \sum_{k=1}^n \lambda_k^{i+j}. \]
Montrer que~$H$ est une matrice symétrique réelle.
Montrer que le rang de~$H$ est égal au nombre de racines distincte, et que~$H$ est positive si et seulement si toutes les racines sont réelles.

\begin{em}
  Il faudrait parler de la signature d'une forme quadratique pour énoncer le
  théorème de manière plus précise, mais c'est hors programme.
  Cette version se passe de la théorie des formes quadratiques.
\end{em}

%%% Local Variables:
%%% mode: latex
%%% TeX-master: "main"
%%% End:


\printbibliography

\end{document}
