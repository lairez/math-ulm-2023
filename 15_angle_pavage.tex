\section{Angles d'un pavage}

Soit $G = \left\{ z\to az+b \st a, b \in \mathbb{C}, |a| = 1 \right\}$. C'est un sous-groupe du groupe des bijections~$\mathbb{C}\to\mathbb{C}$ muni de la composition.
Soit~$H \subseteq G$ un sous-groupe contenant deux translations selon des vecteurs~$b_1,b_2 \in \mathbb{C}$ formant une famille libre sur~$\mathbb{R}$.
On suppose de plus que pour tout~$h \in H$, soit~$h(0) = 0$, soit~$|h(0)| \geq 1$.
Montrer que l'ensemble $\left\{ h'(0) \st h\in H \right\}$ est fini.

Montrer que le cardinal de cet ensemble divise~6.

\begin{em}
  C'est l'un des ingrédients de la classification des pavages du plan.
  L'existence de deux translations n'est pas nécessaire, une seule suffit, mais en donner deux pouvait rendre l'énoncé plus accessible.
\end{em}

%%% Local Variables:
%%% mode: latex
%%% TeX-master: "main"
%%% End:
