\section{Une propriété de divisibilité du cardinal des matrices inversibles modulo~$p$}

Soit $p$ un entier premier impair, et $n\geq 3$ un entier. Montrer que $n$
divise le cardinal du groupe $\operatorname{GL}_{n-1}(\mathbb{Z}/p\mathbb{Z})$ des
matrices inversibles de taille~$n-1$ à coefficients
dans~$\mathbb{Z}/p\mathbb{Z}$.

\begin{em}
  La restriction à~$p$ impair n'est pas nécessaire, mais elle peut simplifier
  certains calculs.
  La formule du cardinal n'est pas au programme mais c'est un classique.
  Pour ceux qui ne la connaissait pas, l'exercice est intéressant.
  On pouvait aussi s'en passer (voir la solution algébrique).
\end{em}

\subsection*{Solution arithmétique}

On sait que le cardinal de $\operatorname{GL}_{n-1}(\mathbb{Z}/p \mathbb{Z})$ est égal à
\[ N = p^{\frac{n (n-1)}{2}} \prod_{i=1}^{n-1}  (p^i  -1). \]
Par le théorème d'Euler, $n$ divise~$p^{\phi(n)} - 1$, où~$\phi(n)$ est l'indicatrice d'Euler.
Comme~$\phi(n) \leq n-1$, $p^{\phi(n)} - 1$ divise~$N$, donc $n$ divise aussi~$N$.

\subsection*{Solution algébrique}

Par le théorème de Lagrange, il suffit de trouver un élément d'ordre~$n$ dans~$\operatorname{GL}_{n-1}(\mathbb{Z}/p\mathbb{Z})$.
On peut prendre~$M_P$, la matrice compagnon du polynôme~$P = 1+X+X^2+ \dotsb + X^{n-1}$.
Le polynôme caractéristique de cette matrice est~$P$ lui-même. Ce polynôme divise~$X^n-1$, donc~$M_P^n = I$.
Comme~$M_P^k \cdot e_1 = e_{k+1}$, pour~$1\leq k\leq n-1$, la matrice n'est pas d'ordre strictement inférieur à~$n$.
Donc elle est d'ordre~$n$.

%%% Local Variables:
%%% mode: latex
%%% TeX-master: "main"
%%% End:
